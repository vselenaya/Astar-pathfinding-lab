\documentclass[12pt, a4paper]{extarticle}
\usepackage[T2A]{fontenc}
\usepackage[left=3cm,right=1cm,
top=2cm,bottom=2cm]{geometry}
\usepackage[utf8]{inputenc}
\usepackage[russian]{babel}
\usepackage{fontspec} 
\setmainfont{Times New Roman}

\usepackage{algorithm}
\usepackage{etoolbox}
\usepackage{algpseudocode}
% Межстрочный интервал = 1.5pt
\usepackage{setspace}
\onehalfspacing

\usepackage{graphicx}
\usepackage{amsmath,amssymb,dsfont,amsthm} % amsthm для proof

% Шрифт подписи (caption) = 12pt
% (Повезло, что small как раз равен 12pt)
\usepackage[font=small,labelfont=bf]{caption}
%\usepackage{print_biblio}  % рукописыный код для настройки отображения списка литературы
\DeclareMathOperator*{\argmin}{arg\,min}
\usepackage{multicol}
\usepackage{titlesec}
\usepackage{titletoc}
\usepackage{tocloft}
\usepackage{changepage}





%%% ==============================================
% для красивых комментариев слева!
\usepackage{tikz}  
\usetikzlibrary{decorations.pathreplacing,calc}
\newcommand{\tikzmark}[1]{\tikz[overlay,remember picture] \node (#1) {};}

\newcommand*{\SpaceReservedForComments}{3.5cm}%
\newcommand*{\HorizontalOffset}{-0.5em}%
\newcommand*{\VerticalOffset}{0.7ex}%
\newcommand*{\AddNote}[4][]{%
    %% #1 = draw options
    %% #2 = top line number to start comment from
    %% #3 = bottom line number where comment ends
    %% #4 = text of comment
    \begin{tikzpicture}[overlay, remember picture]
        \draw [decoration={brace,amplitude=0.5em},decorate,ultra thick,red, #1]
            ($(#3)+(\HorizontalOffset,-\VerticalOffset)$) --  ($(#2)+(\HorizontalOffset,\VerticalOffset)$)
            node [align=left, text width=\SpaceReservedForComments-1.0em, pos=0.5, anchor=east] {\footnotesize #4};
    \end{tikzpicture}
}%

\makeatletter% Add a \tkizmark for each line so we can reference it later
    \algrenewcommand\alglinenumber[1]{\tikzmark{\arabic{ALG@line}}\footnotesize#1:}
\makeatother
%%% ==============================================


% Переопределяем команду \Comment
\usepackage{xcolor}
\definecolor{darkgreen}{RGB}{23, 146, 39}
\renewcommand{\Comment}[1]{\textcolor{darkgreen}{~~~~// #1}}



\begin{document}


\captionsetup[algorithm]{labelsep=colon}
\begin{algorithm}[H]
\hspace*{\SpaceReservedForComments}{} %!!!!!!!!!!!!!
\begin{minipage}{\dimexpr\linewidth-\SpaceReservedForComments\relax}%!!!!!!!!!!!!!
\onehalfspacing  % для 1.5-интервала записи алгоритма
\floatname{algorithm}{}
\caption{Алгоритм A*}
\begin{algorithmic}[1]
\State $\texttt{OPEN} \gets \left\{start\right\}$ \Comment{изначально известно только о старте}
\State $\texttt{CLOSED} \gets \emptyset$  \Comment{пока ни одна вершина не раскрыта, список пуст}
\State ~
\While{$\texttt{OPEN} \neq \emptyset$}
    \State $s \gets \argmin\limits_{v \in \texttt{OPEN}}~f(v)$  \Comment{достаём вершину с минимальным $f$-значением}

    \vspace*{0.2cm}
    
    \State \text{remove $s$ from } \texttt{OPEN}
    \If{$s$ is goal}
        \State \text{\textbf{return} \texttt{PathTo}($s$)}  \Comment{находим и возвращаем путь до $s$}
    \EndIf
    \State ~
    \ForAll{$s' \in \texttt{Successors}(s)$} \Comment{перебираем соседей вершины $s$}
        \If{$s' \notin \texttt{CLOSED}$}  \Comment{уже раскрытые вершины не интересны!}
            \If{$s' \notin \texttt{OPEN}$} 
                \State insert $s'$ into \texttt{OPEN}  \Comment{добавляем новую вершину}
                \State $g(s') \gets g(s) + cost(s, s')$  \Comment{и считаем её $f$,$g$-значения}
                \State $f(s') \gets g(s') + h(s')$
            \ElsIf{$g(s') > g(s) + cost(s, s')$} 
                \State $g(s') \gets g(s) + cost(s, s')$   \Comment{обновляем $f,g$-значения}
                \State $f(s') \gets g(s') + h(s')$
            \EndIf
        \EndIf
    \EndFor
    \State insert $s$ into \texttt{CLOSED}  \Comment{закончили раскрытие $s$, поместили в список}
\EndWhile
\State ~
\State \textbf{return} path not found  \Comment{если не нашли путь, его не существует}
\end{algorithmic}


% Now insert all to comments that we desire on specific line numbers:
\AddNote[blue]{13}{16}{Если $s'$ новая (не встречалась ни в одном списке)}
\AddNote[darkgreen]{5}{6}{На данной итерации раскрываем вершину $s$}
\AddNote[red]{17}{20}{Иначе $s'$ интересна, если нашли до неё путь короче}
\end{minipage}

\end{algorithm}
\end{document}
